\documentclass[a4paper,12pt]{article}

\usepackage{url}
\usepackage{parskip}
\RequirePackage{color}
\RequirePackage{graphicx}
\usepackage[usenames,dvipsnames]{xcolor}
\usepackage[scale=0.875]{geometry}
\usepackage{tabularx}
\usepackage{booktabs}
\usepackage{enumitem}
\newcolumntype{C}{>{\centering\arraybackslash}X}
\usepackage{supertabular}
\newlength{\fullcollw}
\setlength{\fullcollw}{0.47\textwidth}
\usepackage{titlesec}
\usepackage{multicol}
\usepackage{multirow}
\usepackage[unicode, draft=false]{hyperref}
\definecolor{linkcolour}{rgb}{0,0.2,0.6}
\hypersetup{colorlinks,breaklinks,urlcolor=linkcolour,linkcolor=linkcolour}
\usepackage{fontawesome5}

\titleformat{\section}{\large\scshape\raggedright}{}{0em}{}[\titlerule]
\titlespacing{\section}{0pt}{10pt}{10pt}

\begin{document}
\pagestyle{empty}

% HEADER
\noindent\hfill\textbf{October 2025}

\vspace{0.5em}
\begin{center}
{\LARGE \textbf{JUAN LLAVADOR PERALT}}\\[2pt]
\end{center}
\vspace{2em}


\noindent
\begin{tabularx}{\linewidth}{@{}p{0.58\linewidth}@{\hspace{1.5em}}p{0.35\linewidth}@{}}
\begin{minipage}[t]{\linewidth}
\textbf{Institute for International Economic Studies}\\
Stockholm University\\
106 91 Stockholm, Sweden
\end{minipage}
&
\begin{minipage}[t]{\linewidth}
\raisebox{-0.05\height}\faEnvelope\ \href{mailto:juan.llavadorperalt@iies.su.se}{juan.llavadorperalt@iies.su.se}\\
Website: \href{https://juanlla.github.io}{juanlla.github.io}\\
Citizenship: Spanish
\end{minipage}
\end{tabularx}

% EDUCATION
\section*{Education}
\begin{tabularx}{\linewidth}{@{}l X@{}}
2020--2026 & \textbf{Ph.D. in Economics}, IIES -- Stockholm University (expected) \\
2025 & \textbf{Visiting Ph.D. Student}, Harvard University \\
2018--2019 & \textbf{M.Sc. in Economics}, The London School of Economics and Political Science \\
2014--2018 & \textbf{B.Sc. in Economics}, University of Valencia \\
\end{tabularx}

% FIELDS
\section*{Research and Teaching Fields}
Macroeconomics, Firm Dynamics, Quantitative Economics, Distributions in Economics and Finance

% REFERENCES
\section*{References}
\begin{tabularx}{\linewidth}{@{}X X@{}}
\begin{minipage}[t]{\linewidth}
\textbf{Per Krusell}\\
IIES, Stockholm University\\
+46 (0)8 16 30 73\\
\href{mailto:per.krusell@iies.su.se}{per.krusell@iies.su.se}
\end{minipage}
&
\begin{minipage}[t]{\linewidth}
\textbf{Xavier Gabaix}\\
Harvard University\\
+1 (617) 495-2143\\
\href{mailto:xgabaix@fas.harvard.edu}{xgabaix@fas.harvard.edu}
\end{minipage}
\\
\addlinespace[1.5em]
\begin{minipage}[t]{\linewidth}
\textbf{Timo Boppart}\\
UZH and IIES, Stockholm University\\
+46 (0)8 16 35 52\\
\href{mailto:timo.boppart@iies.su.se}{timo.boppart@iies.su.se}
\end{minipage}
&
\begin{minipage}[t]{\linewidth}
\textbf{Joshua Weiss}\\
U.\ of Bristol and IIES, Stockholm University\\
+44 (0)20 795 7508\\
\href{mailto:joshua.weiss@iies.su.se}{joshua.weiss@iies.su.se}
\end{minipage}
\end{tabularx}

% RESEARCH PAPERS
\section*{Research Papers}

\noindent\textbf{Granular Firms and the Concentration Drag on Growth (JOB MARKET PAPER)}\\
This paper studies the dynamic effect of market concentration on productivity growth. I develop a multisector model with granular firms and idiosyncratic productivity shocks and characterize the stochastic dynamics of firms and aggregates. At the sector level, higher concentration lowers future productivity growth by reducing the reallocation gains from idiosyncratic shocks under gross substitutability. I denote this negative effect the concentration drag on growth, which is amplified when firms set markups strategically. At the micro level, granularity generates size-dependent dynamics: small firms are more volatile but have higher growth potential, while the opposite is true for large firms. Using firm- and industry-level data, I provide empirical evidence consistent with these predictions and estimate the model. An increase in concentration due to idiosyncratic shocks is associated with a contemporaneous productivity growth burst, followed by a long-lasting slowdown in productivity growth. Quantitatively, the model predicts that in a typical industry, a 5-percentage-point increase in the Herfindahl index reduces five-year productivity growth by 0.7 percentage points.

\vspace{0.75em}
\noindent\textbf{Inflation Persistence and a New Phillips Curve (with Chek Choi, Marcus Hagedorn, and Kurt Mitman)}\\
Auclert et al. (2024) recently argued that, to first order, menu-costs models deliver the same New Keynesian Phillips Curves as time-dependent models in response to AR(1) shocks. We show here that when considering a broader class of shocks, menu-costs models can generate qualitatively and quantitatively different Phillips curves than implied by time-dependent models. Shocks to the growth rate of nominal demand generate inflation persistence in the model, in line with the data, but at odds with the standard time-dependent NKPC. Changes in the extensive margin of price adjustment in the menu-cost model generate history dependence that is captured by the lagged inflation rate. Once we control for lagged nominal demand growth, the explanatory power of lagged inflation drops significantly. The reason is that nominal demand growth is a second determinant of inflation in the Phillips curve in menu-cost models and inflation therefore inherits the persistence of the process for nominal demand.

\section*{Selected Work in Progress}

\noindent\textbf{Skewed Firm Dynamics}\\
This paper documents a new empirical regularity: the skewness of firm-growth rates declines systematically with firm size. Using Swedish administrative balance sheet data, I show that this pattern is driven by a collapse in the right tail—large firms experience fewer extreme positive growth events, rather than more negative shocks. This finding provides new evidence on why larger firms experience less volatile growth: volatility declines because large positive shocks become rarer, and these shocks have permanent effects on firm size.

\vspace{0.75em}
\noindent\textbf{Industrial Policy with Fat Tails (with Thomas Mikaelsen)}
\vspace{1em}
% PROFESSIONAL ACTIVITIES
\section*{Professional Activities}
\begin{tabularx}{\linewidth}{@{}l X@{}}
2019--2020 & Research Assistant, The Centre for Economic Performance (LSE) \\
\end{tabularx}

% TEACHING EXPERIENCE
\section*{Teaching Experience}
\begin{tabularx}{\linewidth}{@{}l X@{}}
2024 & Lecturer, PhD Mathematics II (Stockholm University) \\
2021--2022 & TA, PhD Macroeconomics I, for Timo Boppart (Stockholm University) \\
2019--2020 & TA, EC210 Intermediate Macroeconomics, for Kevin Sheedy and Ricardo Reis (LSE) \\
\end{tabularx}

% HONORS / SCHOLARSHIPS / FELLOWSHIPS
\section*{Honors, Scholarships, and Fellowships}
\begin{tabularx}{\linewidth}{@{}l X@{}}
2022 & Jan Wallander and Tom Hedelius Foundation (Research Visit) \\
2020 & La Caixa Fellowship for Postgraduate Studies \\
2018 & Premio Extraordinario Fin de Carrera (Best Academic Record, BSc in Economics) \\
2018 & Fundación Ramón Areces Scholarship for Postgraduate Studies \\
\end{tabularx}

\section*{Languages}
Human: Spanish (native), English (fluent), German (fluent), Catalan (intermediate) \\
Computer: Julia, Python, \LaTeX

\vfill
\center{\footnotesize Last updated: \today}

\end{document}